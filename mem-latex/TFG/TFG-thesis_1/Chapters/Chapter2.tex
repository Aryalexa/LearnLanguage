% Chapter Template

\chapter{Diseño e implementación} % Main chapter title

\label{Chapter2} % Change X to a consecutive number; for referencing this chapter elsewhere, use \ref{ChapterX}

En esta sección explicaremos detenidamente los distintos pasos que se llevaron a cabo para obtener la aplicación final. 

Como se ha indicado en la introducción, la tarea fundamental de nuestra aplicación consiste en procesar lo que diga el usuario para poder comparar su pronunciación con la de un nativo. La aplicación está enfocada al tratamiento y comparación del sonido y puede ser incluida por cualquier otra aplicación con interfaz de usuario, ya sea una aplicación móvil, de escritorio u online. Teniendo esto en cuenta se diseñó la aplicación.

Una de las decisiones fue programar en lenguaje Java ya que es con el que tengo más dominio y experiencia. Además, Java tiene muchas bibliotecas y recursos que hemos tratado de aprovechar. El código puede encontrarse REF \cite{REF} % REF
 y es código abierto y libre. La estructura del código de la aplicación se puede encontrar en esta memoria. REF \cite{REF} % REF

Daremos en primer lugar una visión general de todo el programa. Cuando una persona use el programa se grabará diciendo una frase indicada y la aplicación dirá si la pronunciación ha sido suficientemente buena y leal a la de un nativo. Por ello, el primer paso es obtener un audio directamente del micrófono del medio que esté usando el usuario. El sonido capturado se guardará digitalmente como una secuencia de intensidades que contienen toda la información de audio, este contiene muchas frecuencias no útiles para nuestro objetivo de comparar voces pronunciando la misma frase. Por ello habrá una etapa de filtrado. Después se hará una transformación de Fourier a la señal dividida en segmentos, que nos dará el valor de los impulsos por frecuencia de los segmentos. Aprovecharemos que se encuentra en el dominio de la frecuencia para quitar el sonido de fondo. Y con el espectrograma resultante de la transformación limpia nos quedaremos con la matriz bidimensional asociada, donde cada valor representa la intensidad del impulso por tiempo (segmento) y frecuencia. Esta matriz se puede interpretar como una imagen donde el color en cada punto se consigue a partir del elemento correspondiente en la matriz. Una vez que tenemos la imagen, se hace una detección de blobs o regiones para distinguir las zonas más destacables o sobresalientes de la imagen. Con esto conseguiremos una matriz de valores con más contraste y lista para ser comparada con otra matriz de la misma clase, teniendo en cuenta que las dimensiones de las matrices pueden variar ya sea por la velocidad o volumen del hablante.

En el resto de la sección daremos los detalles de estos pasos, dando en primer lugar una explicación teórica y desarrollando después los principales aspectos de implementación.



%----------------------------------------------------------------------------------------
%	SECTION 1
%----------------------------------------------------------------------------------------

\section{Capturando la voz}

Cuando el usuario grabe su voz para ser procesada debemos capturar el audio en un formato que nos facilite su tratamiento. Grabar en formato de audio \emph{raw} nos da total control y visualización de los datos capturados.
Cuando emitimos el sonido para ser grabado emitimos una señal analógica (continua) que el micrófono captura y guarda en forma de señal digital (discreta). 
La transformación de esta señal analógica a digital se hace a través de un proceso llamado muestreo.
En procesamiento de señales, el muestreo es la reducción de una señal continua a una señal discreta. Consiste en tomar muestras de una señal analógica a una frecuencia o tasa de muestreo (sample rate) constante, para cuantificarlas y codificarlas posteriormente. La cuantificación consiste en atribuir un valor finito (discreto) de amplitud a cada muestra, un valor dentro de un conjunto específico de valores que luego es codificado en bits.

%-----------------------------------
%	SUBSECTION 1
%-----------------------------------
\subsection{Subsection 1}

Nunc posuere quam at lectus tristique eu ultrices augue venenatis. Vestibulum ante ipsum primis in faucibus orci luctus et ultrices posuere cubilia Curae; Aliquam erat volutpat. Vivamus sodales tortor eget quam adipiscing in vulputate ante ullamcorper. Sed eros ante, lacinia et sollicitudin et, aliquam sit amet augue. In hac habitasse platea dictumst.

%-----------------------------------
%	SUBSECTION 2
%-----------------------------------

\subsection{Subsection 2}
Morbi rutrum odio eget arcu adipiscing sodales. Aenean et purus a est pulvinar pellentesque. Cras in elit neque, quis varius elit. Phasellus fringilla, nibh eu tempus venenatis, dolor elit posuere quam, quis adipiscing urna leo nec orci. Sed nec nulla auctor odio aliquet consequat. Ut nec nulla in ante ullamcorper aliquam at sed dolor. Phasellus fermentum magna in augue gravida cursus. Cras sed pretium lorem. Pellentesque eget ornare odio. Proin accumsan, massa viverra cursus pharetra, ipsum nisi lobortis velit, a malesuada dolor lorem eu neque.

%----------------------------------------------------------------------------------------
%	SECTION 2
%----------------------------------------------------------------------------------------

\section{Main Section 2}

Sed ullamcorper quam eu nisl interdum at interdum enim egestas. Aliquam placerat justo sed lectus lobortis ut porta nisl porttitor. Vestibulum mi dolor, lacinia molestie gravida at, tempus vitae ligula. Donec eget quam sapien, in viverra eros. Donec pellentesque justo a massa fringilla non vestibulum metus vestibulum. Vestibulum in orci quis felis tempor lacinia. Vivamus ornare ultrices facilisis. Ut hendrerit volutpat vulputate. Morbi condimentum venenatis augue, id porta ipsum vulputate in. Curabitur luctus tempus justo. Vestibulum risus lectus, adipiscing nec condimentum quis, condimentum nec nisl. Aliquam dictum sagittis velit sed iaculis. Morbi tristique augue sit amet nulla pulvinar id facilisis ligula mollis. Nam elit libero, tincidunt ut aliquam at, molestie in quam. Aenean rhoncus vehicula hendrerit.